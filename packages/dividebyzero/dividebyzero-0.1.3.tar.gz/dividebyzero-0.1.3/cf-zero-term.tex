

\section{Continued Fractions with Zero-Term Patterns}
\label{sec:zero-continued-fractions}

\subsection{Motivations in Mathematical Physics}
Continued fractions are pervasive in both mathematics and physics, finding applications in areas ranging from number theory and Diophantine approximations to integrable systems and functional equations. Typically, the presence of \emph{zero terms} in continued fractions (e.g., \([a_0; 0, a_2]\)) is considered an edge case. However, a deeper exploration reveals that incorporating zero terms can lead to \emph{dimensional reduction} phenomena. In our dimensional division framework, these zero terms correspond to singular or boundary conditions that effectively modify the rank or dimensionality of the problem.

More broadly, zero-term continued fractions emerge in:
\begin{itemize}
  \item \textbf{Quantum Field Simplifications}: Here, "singular denominators" can be interpreted through techniques such as dimensional regularization or truncated expansions.
  \item \textbf{Number-Theoretic Expansions}: This includes scenarios with alternating sign expansions or partial quotients that vanish at specific steps.
  \item \textbf{Operator Factorization}: In these cases, an operator factor may become zero and must be substituted with a lower-dimensional projector.
\end{itemize}

\subsection{Focusing on Stability Conditions for Zero Patterns}
Among the four potential research directions:
\begin{enumerate}
  \item Develop specific pattern evolution equations,
  \item Analyze stability conditions,
  \item Explore topological aspects,
  \item Investigate conservation laws,
\end{enumerate}
this section concentrates on \emph{(2)~stability conditions} associated with various zero-term patterns, as detailed in Section~\ref{sec:stability}. The central inquiry is: \emph{Does the introduction of a zero in the continued fraction expansion destabilize or stabilize the dimensional projection process, and how is this reflected in the error norms and projection norms across multiple reductions?}

\subsection{Dimensional Operators and Classification of Zero Patterns}
Building on the framework established in "Mathematical Foundations of Zero-Term Patterns in Extended Continued Fractions" (refer to Listing~\ref{lst:analysis-code} in Appendix~\ref{appendix:code}), we categorize zero patterns into three primary types:
\begin{itemize}
  \item \textbf{Alternating Zero Patterns (A-type)}: Zeros appear at intermittent positions, often in a repeating or interleaved manner.
  \item \textbf{Boundary Zero Patterns (B-type)}: Zeros occur at the very beginning or end of the expansion.
  \item \textbf{Complex Zero Patterns (C-type)}: Multiple or sequential zeros create higher-order interactions, necessitating more intricate multi-step projections.
\end{itemize}
In our implementation, these patterns trigger dimensional reduction operations (via \texttt{\_dimensional\_divide}) each time a zero is detected, resulting in a \emph{projection} and a residual \emph{error} component.

\subsection{Stability Analysis of Zero-Term Expansions}
\label{sec:stability}

\paragraph{Definition of Stability.}
A zero-term configuration is deemed \emph{stable} if the repeated application of the dimensional division operator (along with subsequent partial fraction updates) does \emph{not} lead to unbounded error growth or runaway projection norms. Formally, for a zero pattern generating partial fractions \(\pi_n\) and residuals \(\epsilon_n\):
\[
\text{Stability} \;\iff\; \sup_{n} \|\epsilon_n\| < \infty 
\quad\text{and}\quad 
\sup_{n} \|\pi_n\| < \infty.
\]

\paragraph{Empirical Indicators.}
\begin{enumerate}
  \item \emph{Error Norms} \(\|\epsilon_n\|\): If this sequence remains bounded or tends toward a fixed plateau as zeros are introduced, the configuration is considered stable.
  \item \emph{Projection Norms} \(\|\pi_n\|\): A steady or convergent projection norm indicates robustness, whereas an unbounded increase may signal instability due to poorly placed zeros.
\end{enumerate}

\subsection{Illustrative Examples: Alternating vs. Complex Patterns}
Utilizing the Python code provided in Listing~\ref{lst:analysis-code} within the \verb|DimensionalContinuedFraction| module, we perform analyses on different zero-term expansions. Consider the following examples:
\begin{itemize}
  \item \textbf{Alternating Zero Pattern}: \([1, 0, 2, 0, 3, 0, 4]\).
  \item \textbf{Complex Pattern with Multiple Zeros}: \([1, 2, 0, 3, 0, 0, 4, 5, 0]\).
\end{itemize}
In each case, the code generates plots of the \emph{Error Norm} and \emph{Projection Norm} across each dimensional reduction step. A stable pattern typically exhibits error norms that remain low (e.g., below 1.0 or converging) and projection norms that do not escalate uncontrollably. Conversely, an unstable pattern may display monotonically increasing \(\|\pi_n\|\) or \(\|\epsilon_n\|\).

\subsubsection{Observations from Computational Results}
\begin{itemize}
  \item For \textbf{Alternating Zeros}, there is often \emph{bounded linear growth} in the projection norm (\(\|\pi_n\|\)), accompanied by small incremental errors, indicative of stable expansions.
  \item For \textbf{Complex Zeros} with sequences like \([1, 2, 0, 3, 0, 0, \dots]\), intermittent "jumps" in both projection and error norms may occur. However, in many practical instances, these norms eventually plateau, suggesting an overall stable zero configuration.
\end{itemize}

\paragraph{Intuitive Insights.}
\begin{enumerate}
  \item \emph{Alternating Zeros} distribute the dimensional division operations, resulting in incremental and controlled changes.
  \item \emph{Boundary Zeros} typically introduce minimal error as they effectively trim the continued fraction rather than reorganize it.
  \item \emph{Complex Patterns} may trigger multiple reductions in quick succession, occasionally compounding errors or causing bursts in projection growth. Nonetheless, if each step remains modest, the system can maintain stability overall.
\end{enumerate}

\subsection{Integration into the Overall Framework}
The provided Python code (Listing~\ref{lst:analysis-code} in Appendix~\ref{appendix:code}) implements a systematic approach to detecting and resolving zero terms in a continued fraction:
\begin{enumerate}
    \item Identify indices of zeros.
    \item Handle any leading zeros if present.
    \item Sequentially invoke \texttt{\_dimensional\_divide} on each segment preceding a zero.
    \item Aggregate the resulting \emph{projection} and \emph{error} pairs within a \texttt{DimensionalReduction} structure.
\end{enumerate}
By analyzing the \emph{Error Norms} and \emph{Projection Norms} across each reduction step, one can evaluate the stability of a given zero pattern.

\paragraph{Summary of Stability Insights.}
\begin{itemize}
    \item \textbf{Linear or Bounded Growth} in norms \(\Rightarrow\) the pattern is stable.
    \item \textbf{Exponential or Unbounded Growth} in norms \(\Rightarrow\) the pattern is unstable, though such cases are rare with typical zero placements.
    \item \textbf{Decay or Near-Zero Norms} \(\Rightarrow\) the pattern is extremely stable or represents a trivial expansion, often indicating a dominant leading term.
\end{itemize}

\subsection{Concluding Remarks on Stability}
This analysis demonstrates that zero-term patterns can be systematically classified based on their impact on error and projection norms. Contrary to the assumption that zeros are merely pathological, the \emph{dimensional reduction} perspective uncovers structured and predictable behaviors that can remain stable under iterative processes. These findings have significant implications for:
\begin{itemize}
  \item \textbf{Numerical Implementations}: Enhancing continued fraction expansions that involve singular denominators.
  \item \textbf{Formal Expansions in Operator Algebras}: Managing zero factors corresponding to rank-deficient transformations.
  \item \textbf{Algorithm Design}: Developing algorithms that leverage stable zero placements for partial fraction decomposition or linear algebraic manipulations.
\end{itemize}


\bigskip
\noindent
\textbf{Recommended Next Step.}
If further exploration is desired, one could incorporate \emph{(3) topological aspects} to examine how these zero-based dimensional reductions map onto topological invariants. Alternatively, investigating \emph{(4) conservation laws} can reveal whether certain integral quantities remain unchanged under repeated dimensional division.

\appendix
\section{Code Example}
\label{appendix:code}

\lstset{language=Python,basicstyle=\ttfamily\small,breaklines=true}
\begin{lstlisting}[caption={Enhanced Continued Fraction Analysis Framework},label={lst:analysis-code}]
    """
    Enhanced Continued Fraction Analysis Framework
    ===========================================
    
    This module implements analysis of continued fractions with zero terms through
    dimensional division, with special handling of leading zeros and proper continued
    fraction structure.
    
    Mathematical Framework:
    [a₀; a₁, ..., aₖ] = a₀ + 1/[a₁; ..., aₖ]
    """
    
    import numpy as np
    from typing import List, Tuple, Dict, Optional
    from dataclasses import dataclass
    import matplotlib.pyplot as plt
    from tqdm import tqdm
    
    @dataclass
    class DimensionalReduction:
        """Records dimensional reduction results with enhanced metadata."""
        projection: np.ndarray
        error: np.ndarray
        iteration: int
        zero_position: int
        is_leading_zero: bool = False
    
    @dataclass
    class ContinuedFractionAnalysis:
        """Comprehensive analysis results for continued fractions."""
        original_terms: np.ndarray
        reductions: List[DimensionalReduction]
        final_state: np.ndarray
        total_error: float
        leading_zero_handled: bool = False
    
    class DimensionalContinuedFraction:
        """
        Enhanced continued fraction analysis incorporating proper fraction structure.
        
        Key Features:
        - Leading zero special handling: [0; a₁, ...] = 0 + 1/[a₁; ...]
        - Internal zero dimensional reduction
        - Proper continued fraction structure preservation
        """
        
        def __init__(self, terms: List[float]):
            self.terms = np.array(terms)
            self.zero_indices = [i for i, term in enumerate(terms) if term == 0]
            self.reductions = []
            
        def analyze(self) -> ContinuedFractionAnalysis:
            """
            Analyze continued fraction with proper zero handling.
            
            Mathematical Structure:
            1. Leading zero: Direct addition (0 + 1/rest)
            2. Internal zeros: Dimensional reduction
            3. Trailing zero: Terminal reduction
            """
            # Handle leading zero separately
            current_state = self.terms.copy()
            total_error = 0.0
            iteration = 0
            leading_zero_handled = False
            
            if len(current_state) > 0 and current_state[0] == 0:
                # Leading zero case: a₀ = 0
                leading_zero_handled = True
                if len(current_state) > 1:
                    current_state = current_state[1:]  # Process rest of fraction
                    self.reductions.append(DimensionalReduction(
                        projection=np.array([0.0]),
                        error=np.array([0.0]),
                        iteration=iteration,
                        zero_position=0,
                        is_leading_zero=True
                    ))
                    iteration += 1
            
            # Process remaining zeros
            while 0 in current_state:
                zero_idx = np.where(current_state == 0)[0][0]
                
                # Perform dimensional reduction for non-leading zeros
                if zero_idx > 0 or not leading_zero_handled:
                    projected, error = self._dimensional_divide(
                        current_state[:zero_idx] if zero_idx > 0 else current_state
                    )
                    
                    self.reductions.append(DimensionalReduction(
                        projection=projected,
                        error=error,
                        iteration=iteration,
                        zero_position=zero_idx,
                        is_leading_zero=False
                    ))
                    
                    total_error += np.linalg.norm(error)
                    
                    # Update state
                    if zero_idx < len(current_state) - 1:
                        current_state = np.concatenate([projected, current_state[zero_idx+1:]])
                    else:
                        current_state = projected
                        
                iteration += 1
            
            return ContinuedFractionAnalysis(
                original_terms=self.terms,
                reductions=self.reductions,
                final_state=current_state,
                total_error=total_error,
                leading_zero_handled=leading_zero_handled
            )
        
        def _dimensional_divide(self, partial_fraction: np.ndarray) -> Tuple[np.ndarray, np.ndarray]:
            """Enhanced dimensional division with safety constraints."""
            if partial_fraction.size > 1000:
                raise MemoryError("Input size exceeds safe threshold")
                
            if len(partial_fraction.shape) == 1:
                matrix_form = partial_fraction.reshape(-1, 1)
            else:
                matrix_form = partial_fraction[:1000, :1000]
                
            U, S, Vh = np.linalg.svd(matrix_form, full_matrices=False)
            
            projected = U[:, 0:1] * S[0]
            error = matrix_form - (U[:, 0:1] * S[0] @ Vh[0:1, :])
            
            return projected.flatten(), error
    
    def analyze_and_visualize(test_cases: Dict[str, List[float]]):
        """Enhanced analysis and visualization with leading zero awareness."""
        results = {}
        
        for name, terms in test_cases.items():
            print(f"\nAnalyzing: {name}")
            print(f"Terms: {terms}")
            
            dcf = DimensionalContinuedFraction(terms)
            analysis = dcf.analyze()
            
            print(f"Number of zero terms: {len(dcf.zero_indices)}")
            print(f"Zero positions: {dcf.zero_indices}")
            print(f"Leading zero handled: {analysis.leading_zero_handled}")
            print(f"Total error: {analysis.total_error:.6f}")
            print(f"Final state shape: {analysis.final_state.shape}")
            
            results[name] = analysis
            
            if len(analysis.reductions) > 0:
                plt.figure(figsize=(12, 4))
                plt.subplot(121)
                error_norms = [np.linalg.norm(r.error) for r in analysis.reductions 
                              if not r.is_leading_zero]
                plt.plot(error_norms, '-o')
                plt.title(f"Error Norms: {name}")
                plt.xlabel("Reduction Step")
                plt.ylabel("Error Norm")
                
                plt.subplot(122)
                proj_norms = [np.linalg.norm(r.projection) for r in analysis.reductions 
                             if not r.is_leading_zero]
                plt.plot(proj_norms, '-o')
                plt.title(f"Projection Norms: {name}")
                plt.xlabel("Reduction Step")
                plt.ylabel("Projection Norm")
                
                plt.tight_layout()
                plt.show()
            
        return results
    
    def main():
        """Execute analysis with enhanced test cases."""
        test_cases = {
            "Single Zero": [1, 2, 0, 3, 4],
            "Multiple Zeros": [1, 0, 2, 0, 3],
            "Sequential Zeros": [1, 2, 0, 0, 3],
            "Boundary Zeros": [0, 1, 2, 0],
            "Complex Case": [1, 2, 0, 3, 0, 0, 4, 5, 0],
            "Alternating": [1, 0, 2, 0, 3, 0, 4],
            "Dense Zeros": [1, 0, 0, 1, 0, 0],
            "Large Values": [10, 0, 20, 0, 30]
        }
        
        results = analyze_and_visualize(test_cases)
        
        print("\nSummary Analysis:")
        for name, analysis in results.items():
            print(f"\n{name}:")
            print(f"- Original terms: {analysis.original_terms}")
            print(f"- Number of reductions: {len(analysis.reductions)}")
            print(f"- Leading zero handled: {analysis.leading_zero_handled}")
            print(f"- Total error: {analysis.total_error:.6f}")
    
    if __name__ == "__main__":
        main()
\end{lstlisting}

\begin{figure}[htbp]
    \centering
    \begin{subfigure}[b]{0.45\textwidth}
        \centering
        \includegraphics[width=\textwidth]{single_zero.png}
        \caption{Single Zero}
        \label{fig:single_zero}
    \end{subfigure}
    \hfill
    \begin{subfigure}[b]{0.45\textwidth}
        \centering
        \includegraphics[width=\textwidth]{multiple_zeros.png}
        \caption{Multiple Zeros}
        \label{fig:multiple_zeros}
    \end{subfigure}
\end{figure}

\begin{figure}[htbp]
    \centering
    \begin{subfigure}[b]{0.45\textwidth}
        \centering
        \includegraphics[width=\textwidth]{sequential_zeros.png}
        \caption{Sequential Zeros}
        \label{fig:sequential_zeros}
    \end{subfigure}
    \hfill
    \begin{subfigure}[b]{0.45\textwidth}
        \centering
        \includegraphics[width=\textwidth]{boundary_zeros.png}
        \caption{Boundary Zeros}
        \label{fig:boundary_zeros}
    \end{subfigure}
    
    \begin{subfigure}[b]{0.45\textwidth}
        \centering
        \includegraphics[width=\textwidth]{complex_case.png}
        \caption{Complex Case}
        \label{fig:complex_case}
    \end{subfigure}
    \hfill
    \begin{subfigure}[b]{0.45\textwidth}
        \centering
        \includegraphics[width=\textwidth]{alternating.png}
        \caption{Alternating Zeros}
        \label{fig:alternating_zeros}
    \end{subfigure}
\end{figure}

\begin{figure}[htbp]
    \centering
    \begin{subfigure}[b]{0.45\textwidth}
        \centering
        \includegraphics[width=\textwidth]{dense_zeros.png}
        \caption{Dense Zeros}
        \label{fig:dense_zeros}
    \end{subfigure}
    \hfill
    \begin{subfigure}[b]{0.45\textwidth}
        \centering
        \includegraphics[width=\textwidth]{large_values.png}
        \caption{Large Values}
        \label{fig:large_values}
    \end{subfigure}
\end{figure}
