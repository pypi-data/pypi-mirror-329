\subsection{Task: Freeze-and-Thaw}
\label{sec:FAT}
This task performs freeze-and-thaw (FaT) calculations, for a list of given systems. Basically, this task
performs subsequent frozen-density embedding calculations for the given \ttt{activeSystems}, where in each
cycle the role of the active system changes. \ttt{environmentSystems} will never be set as active.
\subsubsection{Example Input}
\begin{lstlisting}
+task FAT
  act acetone1
  act acetone2
  env water1
  env water2
  +emb
    embeddingMode NADDFUNC
    naddXCFunc PW91
    naddKinFunc PW91k
  -emb
  maxcycles 6
-task
\end{lstlisting}
\subsubsection{Basic Keywords}
\begin{description}
	\item [\texttt{name}]\hfill \\
	Aliases for this task are \ttt{FAT} and \ttt{FaTTask}.
	\item [\texttt{activeSystems}]\hfill \\
	Accepts multiple active systems.
	\item [\texttt{environmentSystems}]\hfill \\
	Accepts multiple environment systems.
	\item [\texttt{sub-blocks}]\hfill \\
	The embedding (\ttt{emb}) and PCM (\ttt{pcm}) settings are added via sub-block in the task settings.
	Prominent settings in the embedding block that are relevant for this task, and their defaults are:
	\ttt{naddXCFunc=BP86}, \ttt{embeddingMode=LEVELSHIFT}.
	\item [\texttt{maxCycles}]\hfill \\
	The maximum number of FaT iterations. By default \ttt{50}.
	\item [\texttt{gridCutOff}]\hfill \\
	A distance cutoff for the integration grid used to calculate the non-additive  energy functional potentials. Negative values equal no cut off. By default \ttt{-1.0}.
	\item [\texttt{convThresh}]\hfill \\
	Convergence criterion for the absolute change of the density matrices w.r.t. FaT cycles. By default \ttt{1.0e-6}
	\item[\texttt{mp2Type}]\hfill \\
	The MP2-type used for the evaluation of the correlation energy of double-hybrid functionals. By default \ttt{DF} is chosen, which uses the density fitting approach specified with \ttt{densfitCorr} in the system block. Other options are \ttt{AO} to evaluate the full two-electron four-center integrals and \ttt{local} to use local MP2.
	\item [\texttt{calculateSolvationEnergy}]\hfill \\
	Calculate the interaction of the first active system with the joined environment and the energy of the first active system. The environment is referenced as a continuum. By default \ttt{false}.
	\item [\texttt{finalEnergyEvaluation}]\hfill \\
	Perform a final energy evaluation on the supersystem grid after converging the freeze-and-thaw. By default \ttt{true}.
  \item[\texttt{calculateUnrelaxedMP2Density}]\hfill \\
  This vector input specifies if unrelaxed MP2 densities are calculated for the active systems. Therefore, \ttt{calculateUnrelaxedMP2Density\{true false\}} will only calculate the unrelaxed MP2 density for the first active system. By default \ttt{\{\}}.
\end{description}
\subsubsection{Advanced Keywords}
\begin{description}
	\item [\texttt{smallSupersystemGrid}]\hfill \\
	If true will use the smallGridAccuracy of the given active system instead of the normal grid accuracy for the supersystem grid. By default \ttt{false}.
	\item [\texttt{extendBasis}]\hfill \\
	Extend the subsystem basis with basis functions centered in the other subsystems. By default \ttt{false}.
	\item [\texttt{basisExtThresh}]\hfill \\
	Overlap threshold for the extension of the subsystem basis. Needs \ttt{extendBasi = true}. By default \ttt{5.0e-2}.
	\item [\texttt{useConvAcceleration}]\hfill \\
	Turn the convergence acceleration (DIIS/Damping) on. By default \ttt{false}.
	\item [\texttt{diisStart}]\hfill \\
	Density RMSD threshold for the start of the DIIS. Needs \ttt{useConvAcceleration = true}. By default \ttt{5.0e-5}.
	\item [\texttt{diisEnd}]\hfill \\
	Density RMSD threshold for the end of the DIIS. Needs \ttt{useConvAcceleration = true}. By default \ttt{1.0e-4}.
	\item [\texttt{keepCoulombCache}]\hfill \\
	The Fock matrix contributions of the passive systems via their Coulomb interaction is not deleted and kept on disk. By default \ttt{false}.
\end{description}