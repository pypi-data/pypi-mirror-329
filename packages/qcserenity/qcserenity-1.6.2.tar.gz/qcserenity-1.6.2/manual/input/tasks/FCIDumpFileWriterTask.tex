\subsection{Task: FCI Dump File Writer}\label{sec:tasks:FCIDumpFileWriter}
This tasks writes Molpro-style FCI dump file to disk. The integrals are calculated using the RI approximation and local
integral fitting.

An FCI dump file provides all two electron and one electron integrals in MO basis.
The orbital index counting starts at 1. The third and fourth index for the one
particle integrals are zero. The file start with (exactly) 4 lines of header, providing
the number of orbitals, electrons, the number of excess alpha electron (MS2), orbital
symmetry, and system symmetry. After the header the integrals are encoded through their
value and their orbital indices.

A typical FCI dump file may look like this:
\begin{lstlisting}
&FCI NORB= 8, NELEC= 10, MS2= 0,
ORBSYM= 1, 1, 1, 1, 1, 1, 1, 1,
ISYM= 1
&END
  8.2371169364e-01     1     1     1     1
  6.5011589197e-02     2     1     1     1
  6.5841092835e-01     2     2     1     1
  1.0997174278e-01     2     1     2     1
 -6.9437938103e-02     2     1     2     2
 ...
  7.7071181164e-01     8     6     0     0
 -6.6624329107e-16     8     7     0     0
 -6.2473727922e+00     8     8     0     0
\end{lstlisting}

\subsubsection{Example Input}
\begin{lstlisting}
 +task FCIDUMP
  act waterA
  env waterB
  onlyValenceOrbitals true
  +emb
      naddXcFunc pbe
      embeddingMode fermi
  -emb
 -task
\end{lstlisting}

\subsubsection{Basic Keywords}
\begin{description}
    \item [\texttt{name}]\hfill \\
    Aliases for this task are \ttt{FCIDUMP} and \ttt{FCIDUMPFILEWRITERTASK}.
    \item [\texttt{activeSystems}]\hfill \\
    Accepts a single active system for which the FCI dump file will be written.
    \item [\texttt{environmentSystems}]\hfill \\
    Accepts multiple environment systems. Their contribution will be included in the one electron integrals.
    \item [\texttt{outputFilePath}]\hfill \\
    The FCI dump output file path.
    \item [\texttt{orbitalRangeAlpha}]\hfill \\
    The orbital range for the alpha orbitals to be written to the FCI dump file.
    This is also the orbitals range, in the case restricted orbitals are used.
    \item [\texttt{orbitalRangeBeta}]\hfill \\
    The orbital range for the beta orbitals to be written to the FCI dump file.
    \item [\texttt{onlyValenceOrbitals}]\hfill \\
    If true, and no orbital ranges are provided, the orbital ranges are assumed to be the range of valence orbitals.
    \item [\texttt{calculateCoreEnergy}]\hfill \\
    If true, the energy of the core electrons is calculated and printed to the (FCI) output file.
    \item[\texttt{sub-blocks}]\hfill \\
    Possible sub-blocks are the embedding (\ttt{emb} with \ttt{naddXCFunc=BP86}).

\end{description}
\subsubsection{Advanced Keywords}
\begin{description}
    \item [\texttt{mullikenThreshold}]\hfill \\
    Prescreening threshold for the auxiliary function to orbital mapping. By default, \ttt{1.0e-4}.
    \item [\texttt{orbitalToShellThreshold}]\hfill \\
    Prescreening threshold for coefficient values. By default \ttt{1.0e-3}.
    \item [\texttt{valenceOrbitalsFromEnergyCutOff}]\hfill \\
    If true, valence orbitals are determined by an energy cut off. If false, tabulated values are used. By default, \ttt{false}.
    \item[\texttt{energyCutOff}]\hfill \\
    Energy cut off to determine core orbitals. By default, \ttt{-5.0}.
    \item[\texttt{virtualEnergyCutOff}]\hfill \\
    Energy cut off to determine virtual valence orbitals. By default, \ttt{+1.0}]
    \item[\texttt{integralSizeCutOff}]\hfill \\
    Integrals smaller than this value are not written to the output file. By default, \ttt{1e-9}.
    \item[\texttt{doiNetThreshold}]\hfill \\
    Prescreening threshold for the orbital coefficients during calculation of differential overlap integrals. By default, 1e-7.
    \item[\texttt{doiIntegralPrescreening}]\hfill \\
    All orbital combinations ik/jl in the integral (ik|jl) are ignored during the integral transformation that have a
    differential overlap smaller than this threshold. By default, \ttt{1e-7}.
\end{description}