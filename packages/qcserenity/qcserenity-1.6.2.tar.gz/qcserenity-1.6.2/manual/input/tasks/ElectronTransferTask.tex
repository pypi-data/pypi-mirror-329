\subsection{Task: Electron Transfer}

This task calculates adiabatic wave functions from the linear combination of any number of quasi-diabatic states.
Those wave functions are used to calculate several charge-transfer properties. Within a two-state approach, the
analytical electronic coupling and  excitation energy are calculated. For multiple quasi-diabatic states, the Hamilton
matrix in the quasi-diabatic basis as well as the energy levels of the adiabatic wave functions are calculated.
Additionally, spin-density distributions can be calculated and printed to cube files. Moreover, atom-wise spin
populations via a Becke-Population analysis can be obtained. Basically, this task combines the FDE-ET, FDE-diab,
and ALMO multi-state DFT formalisms. This task is only implemented for the \ttt{UNRESTRICTED} case, therefore,
prior FaT calculations must also be carried out in spin-unrestricted mode.
\subsubsection{Example Input}
\textbf{FDE-ET:}
\begin{lstlisting}
+task fdeet
  #quasi-diabatic state 1
  act sys1-state1
  act sys2-state1
  #quasi-diabatic state 2
  act sys1-state2
  act sys2-state2
  states {2 2}
  couple {1 2}
  spindensity true
-task
\end{lstlisting}
\textbf{ALMO multi-state DFT:}
{\color{red}IMPORTANT:} The states were calculated with the \ttt{FreezeAndThawTask} and \ttt{embeddingMode ALMO}. If the 
resulting subsystems are added to one system with the \ttt{SystemAdditionTask}, the input simplifies to:
\begin{lstlisting}
+task et
  act state1
  act state2
-task
\end{lstlisting}

\subsubsection{Basic Keywords}
\begin{description}
  \item [\texttt{name}]\hfill \\
  Aliases for this task are \ttt{ELECTRONTRANSFERTASK}, \ttt{ET} and \ttt{FDEET}.
  \item [\texttt{activeSystems}]\hfill \\
  The systems that construct the quasi-diabatic states (they are usually obtained from separate FaT calculations).
  \item [\texttt{states}]\hfill \\
  This vector input defines the quasi-diabatic states. Herein, the number of elements in the vector defines how many
  quasi-diabatic states are used, while each index specifies the number of subsystems contained in this state. All states must have the same number 
  of systems. Here, the ordering of the \ttt{activeSystems} is important! As an example, \ttt{states\{2 2\}} specifies a linear combination of 2 quasi-diabatic states, where the first state is composed of the first two \ttt{activeSystems} and the second state is composed of the next two \ttt{activeSystems}.
  By default, if no vector is given, each system specifies one quasi-diabatic state.
  \item [\texttt{couple}]\hfill \\
  This vector input specifies which of the specified \ttt{states} shall be used in the linear combination.
  Therefore, \ttt{couple\{1 2\}} can be understood as using states 1 and 2. Additionally, \ttt{couple\{1 2 ; 1 3\}}
  can be understood as two separate FDE-ET runs for which in the first run states 1 and 2 are used, whereas in the
  second run states 1 and 3 are used. Using 3 or 4 quasi-diabatic states can be accomplished by specifying
  \ttt{couple\{1 2 3\}} or \ttt{couple\{1 2 3 4\}}, respectively.
  By default, if no vector is given, all states are coupled.
  \item [\texttt{spindensity}]\hfill \\
  Specifies if the spin density of the adiabatic states shall be calculated and printed to cube file. By default \ttt{false}.
  \item [\texttt{spinpopulation}]\hfill \\
  Specifies if atom-wise spin populations are calculated. If \ttt{true}, the spin-density is calculated and subsequently
  used in a Becke population analysis. See also the keyword \ttt{population}. By default \ttt{false}.
  \item [\texttt{population}]\hfill \\
  This vector input specifies which adiabatic states are used for the population analysis if \ttt{spinpopulation true}. By default \ttt{\{0\}}. This corresponds to the calculation of the spin populations for the first adiabatic state.
\end{description}
\subsubsection{Advanced Keywords}
\begin{description}
  \item [\texttt{disjoint}]\hfill \\
  This vector input specifies the usage of the disjoint approximation for FDE-ET/Diab. Therefore, \ttt{disjoint \{1 2\}} corresponds to joining
  the systems 1 and 2 in each state while all other systems are not coupled. By default \ttt{\{\}}.
  \item [\texttt{printContributions}]\hfill \\
  Specifies whether the real-space representations of the diabatic transition-density matrices shall be printed to cube
  files or not. By default \ttt{false}.
  \item [\texttt{diskMode}]\hfill \\
  Specifies whether the transition-density matrices are written to HDF5 file or kept in memory. This is recommended for FDE-ET/diab calculations in which the available memory is insufficient.
  This is often the case when many quasi-diabatic states are coupled of which each state holds a very large transition-density matrix. By default \ttt{false}.
  \item [\texttt{useHFCoupling}]\hfill \\
  Specifies whether the off-diagonal elements of the Hamilton matrix are calculated alternatively using HF exchange contributions. By default \ttt{false}.
  \item [\texttt{coupleAdiabaticStates}]\hfill \\
  Specifies whether the adiabatic states of different runs (see keyword \ttt{couple}) should be coupled again. By default \ttt{false}.
  \item [\texttt{configurationWeights}]\hfill \\
  Specifies weights of the adiabatic states that are coupled. Must have the same dimensions as \ttt{couple}. Requires \ttt{coupleAdiabaticStates}.

\end{description}
