\subsection{Task: Basis-Set Truncation\label{task:truncation}}
Performs a truncation of the basis set of the given system. Only basis functions centered
on dummy atoms are truncated. All others are considered to be the core basis of the
system. Note that this task can manipulate the geometry of the given system such that 
dummy atoms without any basis functions will not survive. This effectively truncates any associated 
fitting basis.

{\color{red}IMPORTANT:} There will be no orbitals available for the system after executing this
task! Only the density matrix is available. This task should always be followed by an SCF-like step!
\subsubsection{Example Input}
\begin{lstlisting}
  +task BST
    act water
    truncAlgorithm NETPOPULATION
  -task
\end{lstlisting}
\subsubsection{Basic Keywords}
\begin{description}
  \item [\texttt{name}]\hfill \\
    Aliases for this task are \ttt{BASISSETTRUNCATIONTASK}, \ttt{BasisSetTask} and \ttt{BST}.
  \item [\texttt{activeSystems}]\hfill \\
    Will use the first active system and truncate the basis set based on the density of the given system.
  \item [\texttt{truncAlgorithm}]\hfill \\
    The algorithm used for basis set truncation. By default the Mulliken-net-population
    algorithm (\ttt{NETPOPULATION}) is taken. This algorithm eliminates basis functions from the active
    system where the sum of all squared occupied orbital coefficients is lower than \ttt{netThreshold}.
    Alternatively the primitive net-population algorithm (\ttt{PRIMITIVENETPOP}) can be requested. This
    algorithm reduces the number of basis functions by the factor \ttt{truncationFactor}.
  \item[\texttt{netThreshold}] \hfill \\
    Mulliken-net-population up to which an basis function centred on dummy atoms is not truncated, requires
    \ttt{truncAlgorithm}=\ttt{NETPOPULATION}. By default \ttt{1e-4}.
  \item[\texttt{truncationFactor}] \hfill \\
    The ratio of basis functions on dummy atoms used in addition ordered by importance in their MnP, requires
    \ttt{truncAlgorithm}=\ttt{PRIMITIVENETPOP}. By default \ttt{0.0}.
\end{description}