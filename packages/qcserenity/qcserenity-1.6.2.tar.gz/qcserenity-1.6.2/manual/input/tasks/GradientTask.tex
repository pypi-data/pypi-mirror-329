\subsection{Task: Gradient}
This task performs a gradient calculation for a given structure. In case of a subsystem gradient calculation, this task will use all active systems for the gradient calculation. Furthermore, all active systems will be included in the Freeze-and-Thaw procedure. When additional environment systems are given, Frozen-Density Embedding gradients are calculated for the active subsystems.
\subsubsection{Example Input}
\begin{lstlisting}
 +task GradientTask
   system water
 -task
\end{lstlisting}
\subsubsection{Basic Keywords}
\begin{description}
\item [\texttt{name}]\hfill \\
  Aliases for this task are \ttt{GradientTask}, \ttt{Gradient} and \ttt{Grad}.
\item [\texttt{activeSystems}]\hfill \\
  Electronically optimizes all active systems. If more than one system is given, they are coupled via Freeze-and-Thaw calculations.
\item [\texttt{environmentSystems}]\hfill \\
  Environment systems are added to Freeze-and-Thaw or Frozen-Density Embedding calculations of the active system(s) but remain electronically frozen and are not optimized.
\item [\texttt{sub-blocks}]\hfill \\
  Embedding (\ttt{emb}) settings are added via sub-blocks in the task settings.
  Prominent settings in the embedding block that are relevant for this task, and their defaults are:
  \ttt{naddXCFunc=PW91}, \ttt{embeddingMode=naddfunc}, \ttt{naddkinfunc=PW91K}.
\item [\texttt{gradType}]\hfill \\
  The type of gradients used. Possible options are analytic evaluation with the keyword \ttt{ANALYTICAL} or numerically (3 pt. scheme) with the keyword \ttt{NUMERICAL}. The default is \ttt{ANALYTICAL}.
\item [\texttt{numGradStepSize}]\hfill \\
  Step size for numerical gradients. The default is \ttt{1.0e-3}. 
\item [\texttt{transInvar}]\hfill \\
  Make gradients translationally invariant. The default is \ttt{false}
\item [\texttt{FDEgridCutOff}]\hfill \\
  A distance cutoff for the integration grid used to calculate the non-additive energy functional potentials. Negative values correspond to no cutoff used. The default is \ttt{-1.0}.
\item [\texttt{FaTmaxCycles}]\hfill \\
  The maximum number of FaT iterations. The default is \ttt{50}.
\item [\texttt{FaTenergyConvThresh}]\hfill \\
  Convergence criterion for the density w.r.t. Freeze-and-Thaw. The default is \ttt{1.0e-6}.
\item [\texttt{print}]\hfill \\
  If \ttt{false} the system-wise gradients are not printed to the output. The default is \ttt{true}.
\item [\texttt{printTotal}]\hfill \\
  Enables the printing of the gradients for all atoms of all active systems. The default is \ttt{false}.
  Set this to \ttt{true} for \textsc{SNF}~\cite{SNF2002} calculations.
\end{description}
