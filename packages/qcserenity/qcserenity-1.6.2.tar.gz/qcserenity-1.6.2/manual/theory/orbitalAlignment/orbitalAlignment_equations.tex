\section{Orbital Alignment Strategies}
\label{sec:OrbitalAlignmentStrategies}
This text documents the orbital alignment strategies available in \textsc{Serenity}. They were
first published in Ref.~(\cite{Bensberg2020}).

The task at hand is to align a set of orbitals $\{|i_L\rangle \}$ to a set of template orbitals
$\{|i_K\rangle\}$. We assume that both sets are of the same size and that each orbital in both
sets correspond to a point in the same partial charge space.
To put this into a simpler phrasing: Each orbital can be characterized by a set of orbital-wise partial charges for
which the charge centers (atoms, atom shells ...) are identical for both orbital sets. These partial charges will be
denoted as $ \{q_{iK}^a\} $ for orbital $|i_K\rangle$ and charge center $a$.

In order to arrive at a scheme that allows us to find a unitary transformation within the orbital set
$\{|i_L\rangle \}$ such that it becomes aligned to the template orbital set, we construct a penalty
function $S_{i}^P$ for each orbital as the difference between the partial charges to the power of an even exponent
$P$.
\begin{align}
  S_i^P = \sum_a(q_{iK}^a - q_{iL}^a)^P.
  \label{eq:similarityMeasure}
\end{align}
As partial charges we use shell-wise Intrinsic Atomic Orbital (IAO) charges~\cite{Knizia2013}, which are given by
\begin{align}
  q_{iL}^a = \langle i_L|\hat{n}_a|i_L\rangle
  \label{eq:IAOCharges}
\end{align}
where $\hat{n}_a$ is the projector on the functions of a shell $a$ of an orthonormal minimal-basis set for the
molecule
\begin{align}
   \hat{n}_a = \sum_{\rho\in a} |\rho\rangle\langle \rho |.
\end{align}
Note that the criterion can be adjusted freely. Partial charges work, but it is possible to
extend the list of criteria (e.g. by the orbital kinetic energy as discussed below).

Analogous to the IAO charges in Eq.~\ref{eq:IAOCharges}, cross charges can be defined as
\begin{align}
  q_{ijL}^a = \langle i_L|\hat{n}_a|j_L\rangle.
  \label{eq:IAOCrossCharges}
\end{align}

We can minimize the total penalty function $\sum_i S_i^P$ by iterative two by two rotations for all orbital pairs
within the set $\{|i_L\rangle \}$ with the constraint that the orbitals stay orthonormal.
This leads to a Lagrangian $L_\mathrm{rot}$ given for two rotated orbitals $|i^{\prime}_L\rangle$
and $|j^{\prime}_L\rangle$ as
\begin{align}
  \begin{split}
    L_\mathrm{rot}&=\sum_a^\mathrm{shells}\left[
      \left(q_{iK}^a-\langle i^{\prime}_L|\hat{n}_a|i_L^{\prime}\rangle \right)^P
    + \left(q_{jK}^a-\langle j^{\prime}_L|\hat{n}_a|j_L^{\prime}\rangle\right)^P
    \right]\\
    |i_L^{\prime}\rangle &= |i_L^{}\rangle \cos(\phi)+|j_L^{}\rangle\sin(\phi)\\
    |j_L^{\prime}\rangle &= |j_L^{}\rangle \cos(\phi)-|i_L^{}\rangle\sin(\phi).\\
  \end{split}
  \label{eq:LagrangianI}
\end{align}
Here, $|i_L\rangle$ and $|j_L\rangle$ are the initial, none-rotated orbitals.

Note that the orthonormality constrain is automatically fulfilled in Eq.~\ref{eq:LagrangianI} by
the definition of the rotated orbitals.

In the following we want to derive the working equations by inserting the definitions for $|i_L^{\prime}\rangle$
and $|j_L^{\prime}\rangle$ into $L_\mathrm{rot}$:
\begin{align}
  \begin{split}
    L_\mathrm{rot} &= \sum_a^\mathrm{shells}\left[
        \left(\underbrace{q_{iiK}^a-\cos^2(\phi)q_{iiL}^a-\sin^2(\phi)q_{jjL}^a-2\cos(\phi)\sin(\phi)q_{ijL}^a}
        _{L_\mathrm{rot}^{ia}}\right)^P\right.\\
        &~~~~~~~+\left.
        \left(\underbrace{q_{jjK}^a-\cos^2(\phi)q_{jjL}^a-\sin^2(\phi)q_{iiL}^a+2\cos(\phi)\sin(\phi)q_{ijL}^a}
        _{L_\mathrm{rot}^{ja}}\right)^P\right]\\
                   &=\sum_a^\mathrm{shells}\left[
                   \left(L_\mathrm{rot}^{ia}\right)^P+ \left(L_\mathrm{rot}^{ja}\right)^P
                   \right]
  \end{split}
\end{align}
With the elements of the sum $L_\mathrm{rot}^{ia}$ and $L_\mathrm{rot}^{ja}$ we can write the derivative of
$L_\mathrm{rot}$ with respect to the angle $\phi$ as
\begin{align}
  \frac{\partial L_\mathrm{rot} }{\partial \phi} &= \sum_a P \left[
        \left(L_\mathrm{rot}^{ia}\right)^{P-1} \frac{\partial L_\mathrm{rot}^{ia}}{\partial \phi}+
        \left(L_\mathrm{rot}^{ja}\right)^{P-1} \frac{\partial L_\mathrm{rot}^{ja}}{\partial \phi}
                                                   \right]
\end{align}
with
\begin{align}
  \frac{\partial L_\mathrm{rot}^{ia}}{\partial \phi} = 2\cos(\phi)\sin(\phi)(q_{iiL}^a-q_{jjL}^a)
                                                     +2[\sin^2(\phi)-\cos^2(\phi)]q_{ijL}^a
\end{align}
and find
\begin{align}
  \begin{split}
    \frac{\partial L_\mathrm{rot}^{ja}}{\partial \phi} &= -2\cos(\phi)\sin(\phi)(q_{iiL}^a-q_{jjL}^a)
                                                         -2[\sin^2(\phi)-\cos^2(\phi)]q_{ijL}^a\\
                                                       &= -\frac{\partial L_\mathrm{rot}^{ia}}{\partial \phi}.
  \end{split}
\end{align}
We can use these definitions to write the second derivative in a compact way as
\begin{align}
  \begin{split}
    \frac{\partial^2 L_\mathrm{rot}}{\partial \phi^2} &= \sum_a P\left[
         (P-1)\left[\left(L_\mathrm{rot}^{ia} \right)^{P-2}+\left(L_\mathrm{rot}^{ja} \right)^{P-2}\right]
         \left(\frac{\partial L_\mathrm{rot}^{ia}}{\partial \phi}\right)^2\right.\\
        &\left.~~~~~~~~+\left[\left(L_\mathrm{rot}^{ia}\right)^{P-1}-\left(L_\mathrm{rot}^{ja}\right)^{P-1}\right]
         \frac{\partial^2 L_  \mathrm{rot}^{ia}}{\partial \phi^2}
    \right].
  \end{split}
\end{align}
The second derivative of the sum element can be calculated as
\begin{align}
  \begin{split}
    \frac{\partial^2 L_\mathrm{rot}^{ia}}{\partial \phi^2}
     &= 2(q_{iiL}^a-q_{jjL}^a)\left[\cos^2(\phi)-\sin^2(\phi)\right]+8\cos(\phi)\sin(\phi)q_{ijL}^a\\
     &= -\frac{\partial^2 L_\mathrm{rot}^{ja}}{\partial \phi^2}.
  \end{split}
\end{align}

The optimal value of $\phi$ is determined by numerical minimization, which allows for a flexible implementation.
The computational cost of the procedure is negligible and in the same order as for a conventional orbital
localization using the Intrinsic Bond Orbital (IBO)~\cite{Knizia2013} approach.

\subsection{On the Calculation of the Orbital Kinetic Energies}

One of the major benefits of the IBO approach is that all orbital rotations are performed in the IAO
minimal-basis set $\{|\rho\rangle\}$. Since this minimal basis set spans the space of the occupied orbitals
$\{|i_L\rangle\}$ we can write the occupied orbitals as
\begin{align}
  |i_L\rangle = \sum_\rho c^{\mathrm{IAO}}_{\rho i,L} |\rho\rangle,
\end{align}
where $c^{\mathrm{IAO}}_{\rho i,L}$ are the expansion coefficients of the occupied orbitals in the IAO
minimal-basis set. We can calculate the orbital kinetic energy $t_{ii}$ as
\begin{align}
  t_{ii}=\langle i | \hat{t}| i \rangle =
   \sum_{\rho \sigma} c^\mathrm{IAO}_{\rho i,L} c^\mathrm{IAO}_{\sigma i,L} \langle \rho |\hat{t}|\sigma\rangle.
\end{align}
If we precalculate the integrals $\langle \rho |\hat{t}|\sigma\rangle$ we can perform the evaluation of the
kinetic energy contribution to the Lagrangian in the IAO basis as well. Furthermore, this reduces the computational
cost of the kinetic energy evaluation (in the alignment procedure) massively, since the IAO minimal-basis set will
usually be smaller than the actual atomic-basis set.

\subsection{Developers Note}
Exponents of $P=4$ and $P=2$ work well for the penalty function $S_{i}^P$ (eq. \ref{eq:similarityMeasure}). If the exponent becomes to large (e.g. $P=16$), the numerical
minimization of the Lagrangian breaks down and the resulting orbitals resemble the initial canonical
orbitals. I was unable to tell the difference looking at cube files between IBO localized and aligned
orbitals for a water molecule for an exponent of $P=4$ which is equal to the exponent used in the IBO approach.
