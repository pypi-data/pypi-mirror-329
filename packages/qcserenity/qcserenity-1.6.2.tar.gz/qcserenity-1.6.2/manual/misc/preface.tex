\chapter{Preface}
Subsystem and quantum embedding approaches are currently among the most quickly developing
methods in quantum chemistry. This is mostly due to their computational efficiency
and their easy interpretability in terms of intuitive, well-defined chemical subunits.
While the general idea of combining electronic-structure methods of different or the same
types for separate fragments is maybe straightforward, the practical execution of such
calculations can be a frustrating experience. The background is that such a combination
may require interfacing of different codes with limited interoperability through text files.
This strategy can be very successful, but usually also causes an overhead of work for the specific
interfaces needed. Moreover, it is vulnerable to (even slight) changes in the input/output
structure or working principles of the original programs. Even if all technical challenges
are dealt with, inexperienced users may find it hard to choose the optimum embedding approach
for a problem at hand, since comparison of different schemes can be very difficult.\\
\\
This was our motivation for developing a new quantum chemistry code that should give access
to a wide variety of quantum embedding approaches. This should allow for easy comparisons
of embedding schemes, as \serenity is written with a subsystem structure in mind right from
the beginning. Hence, it includes several possibilities for embedding calculations with
different ingredients. Its modular structure should make it easily extensible to upcoming
embedding features. As an open-source project, it may be tailored to user-specific requirements
whenever desired.\\
\\
The following manual shows how to use \serenity as a standalone program and via the Python wrapper. Its usage as a
library is best explained through the source code (detailed manual
information may follow at a later point).
